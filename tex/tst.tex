
\documentclass[12pt]{article}
\usepackage{mausmathex}     % remove clutter -> most things in .sty

\usepackage{fancyhdr}       % for page numbers in a footer
\usepackage{lastpage}

% homework template to be used with mausmathex
% compile with e.g. latexmk --xelatex or lualatex

%suppress numbering for subsections
\setcounter{secnumdepth}{0}

% if we need a bibliography uncomment the following
%\usepackage[hyperref=true, style=alphabetic]{biblatex}

% quick hack for a more sensible article entries in bib
% consider more elaborate biblatex style file as in
%%http://tex.stackexchange.com/questions/10682/suppress-in-biblatex
%\renewbibmacro{in:}{%
%  \ifentrytype{article}{}{\printtext{\bibstring{in}\intitlepunct}}}
%\addbibresource{bibfile.bib}  % biblatex .bib file

% Set these useful commands...
\newcommand{\Student}{Firstname Surname}
\newcommand{\StudentId}{00000000}
\newcommand{\Email}{first.surname@cs.helsinki.fi}
\newcommand{\CourseName}{Introduction to Whatever \hfill Autumn 2015}
\newcommand{\ProblemSet}{Problem Set 0}


\begin{document}
    % fancyhdr + lastpage: page numbering
    \pagestyle{fancyplain}
    \fancyhf{}
    \cfoot{\thepage \, (\pageref{LastPage})}
    \renewcommand{\headrulewidth}{0pt}

    % ... they are used in this frankenarticle title / top matter thing.
    % (TODO: impelementation -> to the package file)

    \noindent
    {\bf \CourseName}

    \noindent
    Homework report for
    {\bf \ProblemSet}

    \bigskip

    {\raggedright
        {\bf \Student} \\
        {\texttt \Email } \\
        Student id \StudentId \\
        \today
    }


    % for short-ish homework sets separate page for ToC seems excessive
    % so we stick it on the same page with the top matter
    \tableofcontents

    \newpage

    % main matter

    \section{Pen-and-paper problems}
    \label{sec:pen_and_paper_problems}


    \subsection{Problem 1}
    \label{sub:problem_1}

    Blah blah \dots

    \paragraph{Part (a)}
    Demonstrate answering to multiparted questions.

    \paragraph{Part (b)}
    Like this.

    \section{Programming problems}
    \label{sec:programming_problems}



\end{document}
